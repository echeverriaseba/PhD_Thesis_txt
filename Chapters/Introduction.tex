\chapter*{\Huge Introduction}
\thispagestyle{thesis} % Force the thesis page style
\markboth{\MakeMarkcase{Introduction}}{\MakeMarkcase{Introduction}} % Set headers
\addcontentsline{toc}{chapter}{Introduction} % Add to Table of Contents

\section*{Agriculture and global change}

% Par. 1: Agriculture and climate change
Interactions between changes in Earth's biogeophysical systems and the effects of human activities are referred to as \textit{global change} \citep{national2001global}. From a climate perspective, food production is both a contributor to such changes, as agriculture is a main driver of anthropogenic greenhouse gas (GHG) emissions \citep{lenka2015, chataut2023}, and an activity prone to its effects, as millions of people have already been exposed to acute food insecurity as a consequence of increasing weather and climate extreme events \citep{fao2022b}. Methane (CH$_{4}$) and nitrous oxide (N$_{2}$O) are the most important GHG, after carbon dioxide (CO$_{2}$), with global warming potentials (GWP) of 27 and 273 higher than CO$_{2}$, respectively \citep{IPCC2021}. Agriculture is the largest source of CH$_{4}$ and its high fertilizer inputs have lead to a steep increase in N$_{2}$O emissions \citep{IPCC2007}. Farming practices, transportation of food products and consumption habits are major agricultural-related determinants of climate change \citep{balogh2020}.\\ % carbon sequestration

% Par. 2: Narrow focus on Global Change - Agriculture and biodiversity.
Even though the critical importance of achieving the agreed 1.5 \degree C limit to global temperature increases regarding pre-industrial levels \citep{agreement2015paris}, focusing solely on global warming narrows the broader analysis of anthropogenic global change, as it is just one of its components \citep{vitousek1994}. Agriculture is implicated in further global effects through extensive land-use changes and variations in ecosystem functioning \citep{tomich2011}. Worldwide, agricultural areas are transitioning towards management intensification and agroecosystem simplification, resulting in soil erosion, reduced organic matter and decreased plant and faunal biodiversity \citep{zimmerer2010}. Intensification, through increasing specialization in production processes, leads to changes in community structure of crop associated biota, such as birds \citep{donald2006further, jeliazkov2016}, fish, amphibians \citep{gopel2020}, macroinvertebrates \citep{perez2023enhanced}, and soil invertebrates and microorganisms \citep{matson1997}. The relation between biodiversity and functioning of agricultural systems has long been recognized \citep{Swift1996-qx}. Besides agriculture contribution to climate change through GHG emissions, multiple key ecosystem functions such as nutrient cycling, biological pest control, water and soil conservation are regulated by the level of functional biodiversity \citep{nicholls1998}. Ultimately, lower crop yields can result form decreased biodiversity due to agriculture intensification and habitat loss \citep{richards2001}.\\   % check Zimmerer; Tomich; Newbold; Matson; Crassman; Dudley

% Par. 3: Projections. Scenarios. Increase of population and CC perspectives linked to it. Importance of biodiversity in agriculture (agrobiodiversity) and its link with ES. How practices might affect biod and ES.  Importance of developing mitigation/adaptation practices (e.g. SDGs, CSA, NBS)


According to the latest \cite{UN2024} projections, the world's population is expected to continue growing up to a peak of 10.3 billion in the mid-2080s, up from 8.2 billion in 2024. As a consequence, food demand is also expected to increase, challenging current agricultural systems and leading to further land-use changes and management intensification \citep{rogelj2018scenarios}. Under this scenario, the most severe global risks perceived for the next ten years are failure to mitigate and adapt to climate change, natural disasters and extreme weather events, and biodiversity loss and ecosystem collapse \citep{worldeconomicforum.2023}. Facing such challenges has driven towards the establishment, promotion, and monitoring of the United Nation's Sustainable Development Goals (SDGs, \cite{un2015transforming}). Frameworks aligned with these goals are, among others, those of Climate-Smart Agriculture (CSA, \cite{fao2011climate}) and Nature-based Solutions (NbS, \cite{noauthor_2020-su}). Even though these include guidelines for practices to achieve a more sustainable production encouraging climate change mitigation and adaptation, enhancing biodiversity and ecosystem functioning, their implementation may lead to undesired trade-offs in between these objectives \citep{seddon2019, smith2022, tripathi2022}. Conflicting outcomes must be avoided through proper assessments of agriculture and ecosystem management and planning, considering joint courses of action for climate change mitigation and biodiversity conservation \citep{rusch2022}.\\ %SDGs
 
%\section*{Wetlands and their ecological role}
\section*{Rice paddies: semi-natural wetlands and source of staple food}
% Par. 1: Rice importance I: (i) Food; (ii) Climate change

Within the current and projected global change scenarios, special care should be put in the assessment of agricultural practices regarding the production of crops that are related to both climate change and biodiversity, while playing important roles in global food security. Rice (\textit{Oryza sativa}) has the third largest area harvested worldwide (over 168.3 million hectares), behind wheat and corn \citep{fao2023a}, and is the most important staple food for over half of the world's population \citep{seck2012crops}. Additionally, 

% name SDGs, CSA and NbS related to rice.
% Rice expansion, replacing previous wetlands.
% Types of rice cultivation per region. Upland, tropical, pemperate continuous flooding.


\section*{Broader role of worldwide rice paddies}
% Par. 1: Rice importance II: (iii) Biodiversity; (iv) System functionality.



\section*{Water-saving irrigation strategies}
% Par. 1: Water saving irrigation practices.
% History.
% Benefits (identified trade-offs?).
% Current implementation (maybe a map here).

\section*{Objectives and structure}
% Par. 1: Objectives and structure



\section*{Study system: The Ebro Delta}
% Par. 6: Study system


